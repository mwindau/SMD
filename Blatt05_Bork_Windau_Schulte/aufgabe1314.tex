\documentclass[
titlepage=firstiscover,
bibliography=totoc,
captions=tableheading,
]{scrartcl}

\usepackage[aux]{rerunfilecheck}



\usepackage{polyglossia}
\usepackage[autostyle]{csquotes}
\setmainlanguage{german}
\setotherlanguages{english, french}

\usepackage{microtype}

\usepackage{amsmath}
\usepackage{amssymb}
\usepackage{mathtools}

\usepackage{fontspec}

\usepackage[
  style=alphabetic,
]{biblatex}
\addbibresource{main.bib}

\usepackage[
math-style=ISO,
bold-style=ISO,
sans-style=italic,
nabla=upright,
partial=upright,
]{unicode-math}

\usepackage[
  locale=DE,
  separate-uncertainty=true,
  per-mode=symbol-or-fraction,
]{siunitx}

\sisetup{
locale=DE,
per-mode=symbol-or-fraction}

\usepackage[unicode]{hyperref}
\usepackage{bookmark}

\usepackage{graphicx}
\graphicspath{{build/}}

\usepackage{caption, booktabs}

\usepackage{grffile}
\usepackage{subcaption}
\usepackage{float}

\usepackage{xfrac}



\title{SMD: Blatt 5}
\author{
  Sophie Bork
  \texorpdfstring{
    \\
    \href{mailto:sophie.bork@udo.edu}{sophie.bork@udo.edu}
  }{}
  \texorpdfstring{\and}{, }
  Simon Schulte
  \texorpdfstring{
    \\
    \href{mailto:simon.schulte@udo.edu}{simon.schulte@udo.edu}
  }{}
  \texorpdfstring{\and}{, }
  Michael Windau
  \texorpdfstring{
    \\
    \href{mailto:michael.windau@udo.edu}{michael.windau@udo.edu}
  }{}
}
\publishers{TU Dortmund – Fakultät Physik}

\begin{document}
  \maketitle


  \section{Aufgabe 13}
    \subsection{Aufgabe 13a}

    Zunächst wird die Transformationsmethode auf den Neutrinofluss angewendet:
    \begin{align*}
      E &\coloneq \frac{E}{\symup{TeV}}\\
      \phi &= \phi_0\cdot(E)^{-\gamma}\\
      \gamma' &\coloneq -\gamma\\
      \implies\,\,\, 1&=\int_1^\infty \phi_0E^{+\gamma'}\,dE=
      \left[\phi_0\frac1{\gamma'+1}E^{\gamma'+1}\right]_1^\infty=-\phi_0
      \frac1{\gamma'+1}\\
      \iff \phi_0&=-(\gamma'+1)\\
      \,\\
      F&=-(\gamma'+1)\int_1^E E^{\gamma'}\,dE \\
      &=-(\gamma'+1)\left[\frac1{\gamma'+1}E^{\gamma'+1}\right]_1^E\\
      &=-(\gamma'+1)\left[\frac1{\gamma'+1}\cdot E^{\gamma'+1}\right]_1^E\\
      &=-(\gamma'+1)\left(\frac1{\gamma'+1}\cdot E^{\gamma'+1}-\frac1{\gamma'+1}\right)\\
      &=-E^{\gamma'+1}+1\\
      &=1-E^{-1,7}=u\\
      \implies &E^{-1,7}=1-u \iff E=(1-u)^{-\frac1{1,7}}
    \end{align*}
    Anschließend werden aus den gleichverteilten Zufallszahlen $u$ $10^5$ Signalereignisse
    simuliert. Diese werden unter dem key "Energy" in das DataFrame "data" gespeichert.

    \subsection{Aufgabe 13b}
    Aus der Wahrscheinlichkeit für eine Detektion wird die Detektorakzeptanz simuliert.
    Für die Signale aus a) wird die Akzeptanz berücksichtigt. Ein Vergleich ist im Plot \ref{fig:plothist}
    zu sehen.
    \begin{figure}[H]
      \centering
      \includegraphics[height=10cm]{plothist.pdf}
      \caption{Histogramm für die Signale mit und ohne berücksichtigter Akzeptanz.}
      \label{fig:plothist}
    \end{figure}

    \subsection{Aufgabe 13c}
    Mittels der Polarmethode werden aus zwei gleichverteilten Zufallszahlen zwei normalverteilte simuliert.
    Diese werden über
    \begin{align*}
      x &= \sqrt{1-\rho^2}\cdot\sigma x^{*}\, +\,\rho\sigma y^{*}\,+\,\mu\\
      y &= \sigma y^{*}\,+\,\mu\\
    \end{align*}
    in die gewünschte Normalverteilung tranformiert.
    Aus dieser Normalverteilung werden die Anzahlen der verschiedenen Hits simuliert.
    Das Ergebniss wird Anschließend im DataFrame unter dem key "NumberOfHits" gespeichert.

    \subsection{Aufgabe 13d}
    Mittels zwei Normalverteilungen mit dem angegebenen $\sigma$ und den $\mu_x = 7$ bzw.
    $\mu_y=3$, wird die Ortsmessung auf dem Flächendetektor simuliert. Aus den Normalverteilungen werden
    die jeweiligen Orte der Hits gezogen, und unter den keys "x" und "y" im DataFrame gespeichert.
    Die Orte sind in Abbildung \ref{fig:detektor} dargestellt.
    \begin{figure}[H]
      \centering
      \includegraphics[height=10cm]{detektor.pdf}
      \caption{Zweidimensionales Histogramm der Ortssimulationen.}
      \label{fig:detektor}
    \end{figure}

    \subsection{Aufgabe 13e}
    Zum Schluss wird der Untergrund simuliert.
    Die Ergebnisse dieser Untergrundsimulation sind in Abbildung \ref{fig:detektoruntergrund} und
    \ref{fig:histuntergrund} dargestellt.
    \begin{figure}[H]
      \centering
      \includegraphics[height=10cm]{detektoruntergrund.pdf}
      \caption{Zweidimensionales Histogramm der Ortssimulationen des Untergrunds.}
      \label{fig:detektoruntergrund}
    \end{figure}
    \begin{figure}[H]
      \centering
      \includegraphics[height=10cm]{histuntergrund.pdf}
      \caption{Histogramm der Untergrundsimulation.}
      \label{fig:histuntergrund}
    \end{figure}





  \section{Aufgabe 14}
    \subsection{Aufgabe 14a}
    Scatterplot für die ersten zwei Dimensionen des Datensatzes:
    \begin{figure}[H]
      \centering
      \includegraphics[height=10cm]{a_scatter.pdf}
      \caption{Scatterplot von $x_1$ und $x_2$.}
      \label{fig:ascatter}
    \end{figure}

    \subsection{Aufgabe 14b}
    Die Hauptkomponentenanalyse sucht nach einer Basis im Raum indem die Varianz entlang der Basisvektoren maximiert wird.\\
    Gegeben seien also N
    Datenpunkte mit d Dimensionen.

    \begin{enumerate}
      \item Zentrierung
        \begin{enumerate}
          \item Mittelwertvektor $\mu$ bilden.
          \begin{align*}
            \mu =
            \begin{pmatrix*}[c]
              \bar{x_1}\\
              \bar{x_2}\\
              \bar{x_3}\\
              \bar{x_4}\\
            \end{pmatrix*}
          \end{align*}
          \item $x_i\, = x_i - \mu$
        \end{enumerate}
      \item Kovarianz
        \begin{enumerate}
          \item Kovarianzmatrix Cov($\symbf{X}$) bilden
        \end{enumerate}
      \item Eigenwerte und Vektoren
        \begin{enumerate}
          \item Die 4 Eigenwerte und Eigenvektoren von Cov($X$) bestimmen, und der Größe nach sortieren.
        \end{enumerate}
      \item Transformierung
        \begin{enumerate}
        \item  Den Datensatz $\symbf{X}$ mit der Transformationsmatrix $\symbf{W}$ aus den        Eigenvektoren multiplizieren.
          \begin{align*}
            \symbf{X}\' = \symbf{X}\symbf{W}
          \end{align*}
          Es ergibt sich die transformierte Matrix $\symbf{X}\'$
        \end{enumerate}
    \end{enumerate}

    \subsection{Aufgabe 14c}
    Die Eigenwerte der Kovarianzmatrix ergeben sich zu:
    \begin{align*}
      \lambda_1 &= 17.519\\
      \lambda_2 &= 0.999\\
      \lambda_3 &= 0.988\\
      \lambda_4 &= 0.899\\
    \end{align*}
    Es ist deutlich dass der erste Eigenwert eine wesentlich höhere Korrelation als
    die anderen beschreibt.

    \subsection{Aufgabe 14d}
    \begin{figure}[H]
      \centering
      \includegraphics[height=7cm]{d_histx1.pdf}
      \caption{Histogramm von $x_1\'$.}
      \label{fig:dhistx1}
    \end{figure}
    \begin{figure}[H]
      \centering
      \includegraphics[height=7cm]{d_histx2.pdf}
      \caption{Histogramm von $x_2\'$.}
      \label{fig:dhistx2}
    \end{figure}
    \begin{figure}[H]
      \centering
      \includegraphics[height=7cm]{d_histx3.pdf}
      \caption{Histogramm von $x_3\'$.}
      \label{fig:dhistx3}
    \end{figure}
    \begin{figure}[H]
      \centering
      \includegraphics[height=7cm]{d_histx4.pdf}
      \caption{Histogramm von $x_4\'$.}
      \label{fig:dhistx4}
    \end{figure}
    \begin{figure}[H]
      \centering
      \includegraphics[height=7cm]{d_scatterx1x4.pdf}
      \caption{Scatterplot von $x_1\'$ und $x_2\'$.}
      \label{fig:dscatter}
    \end{figure}

\end{document}
