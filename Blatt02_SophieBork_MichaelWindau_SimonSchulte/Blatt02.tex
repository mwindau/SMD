\documentclass[
  bibliography=totoc,     % Literatur im Inhaltsverzeichnis
  captions=tableheading,  % Tabellenüberschriften
  titlepage=firstiscover, % Titelseite ist Deckblatt
]{scrartcl}

% Paket float verbessern
\usepackage{scrhack}

% Warnung, falls nochmal kompiliert werden muss
\usepackage[aux]{rerunfilecheck}

% unverzichtbare Mathe-Befehle
\usepackage{amsmath}
% viele Mathe-Symbole
\usepackage{amssymb}
% Erweiterungen für amsmath
\usepackage{mathtools}

% Fonteinstellungen
\usepackage{fontspec}
% Latin Modern Fonts werden automatisch geladen
% Alternativ:
%\setromanfont{Libertinus Serif}
%\setsansfont{Libertinus Sans}
%\setmonofont{Libertinus Mono}
\recalctypearea % Wenn man andere Schriftarten gesetzt hat,
% sollte man das Seiten-Layout neu berechnen lassen

% deutsche Spracheinstellungen
\usepackage{polyglossia}
\setmainlanguage{german}


\usepackage[
  math-style=ISO,    % ┐
  bold-style=ISO,    % │
  sans-style=italic, % │ ISO-Standard folgen
  nabla=upright,     % │
  partial=upright,   % ┘
  warnings-off={           % ┐
    mathtools-colon,       % │ unnötige Warnungen ausschalten
    mathtools-overbracket, % │
},                       % ┘
]{unicode-math}

% traditionelle Fonts für Mathematik
\setmathfont{Latin Modern Math}
% Alternativ:
%\setmathfont{Libertinus Math}

\setmathfont{XITS Math}[range={scr, bfscr}]
\setmathfont{XITS Math}[range={cal, bfcal}, StylisticSet=1]

% Zahlen und Einheiten
\usepackage[
locale=DE,                   % deutsche Einstellungen
separate-uncertainty=true,   % immer Fehler mit \pm
per-mode=symbol-or-fraction, % / in inline math, fraction in display math
]{siunitx}

% chemische Formeln
\usepackage[
version=4,
math-greek=default, % ┐ mit unicode-math zusammenarbeiten
text-greek=default, % ┘
]{mhchem}

% richtige Anführungszeichen
\usepackage[autostyle]{csquotes}

% schöne Brüche im Text
\usepackage{xfrac}

% Standardplatzierung für Floats einstellen
\usepackage{float}
\floatplacement{figure}{htbp}
\floatplacement{table}{htbp}

% Floats innerhalb einer Section halten
\usepackage[
section, % Floats innerhalb der Section halten
below,   % unterhalb der Section aber auf der selben Seite ist ok
]{placeins}

% Seite drehen für breite Tabellen: landscape Umgebung
\usepackage{pdflscape}

% Captions schöner machen.
\usepackage[
  labelfont=bf,        % Tabelle x: Abbildung y: ist jetzt fett
  font=small,          % Schrift etwas kleiner als Dokument
  width=0.9\textwidth, % maximale Breite einer Caption schmaler
]{caption}
% subfigure, subtable, subref
\usepackage{subcaption}

% Grafiken können eingebunden werden
\usepackage{graphicx}
% größere Variation von Dateinamen möglich
\usepackage{grffile}

% schöne Tabellen
\usepackage{booktabs}

% Verbesserungen am Schriftbild
\usepackage{microtype}

% Literaturverzeichnis
\usepackage[style=alphabetic,]{biblatex}
% Quellendatenbank
\addbibresource{lit.bib}
\addbibresource{programme.bib}

% Hyperlinks im Dokument
\usepackage[
  unicode,        % Unicode in PDF-Attributen erlauben
  pdfusetitle,    % Titel, Autoren und Datum als PDF-Attribute
  pdfcreator={},  % ┐ PDF-Attribute säubern
  pdfproducer={}, % ┘
]{hyperref}
% erweiterte Bookmarks im PDF
\usepackage{bookmark}

% Trennung von Wörtern mit Strichen
\usepackage[shortcuts]{extdash}

\title{SMD: Blatt 2}
\author{
  Sophie Bork
  \texorpdfstring{
    \\
    \href{mailto:sophie.bork@udo.edu}{sophie.bork@udo.edu}
  }{}
  \texorpdfstring{\and}{, }
  Simon Schulte
  \texorpdfstring{
    \\
    \href{mailto:simon.schulte@udo.edu}{simon.schulte@udo.edu}
  }{}
  \texorpdfstring{\and}{, }
  Michael Windau
  \texorpdfstring{
    \\
    \href{mailto:michael.windau@udo.edu}{michael.windau@udo.edu}
  }{}
}
\publishers{TU Dortmund – Fakultät Physik}


\begin{document}

\maketitle
\thispagestyle{empty}
\newpage
\setcounter{page}{1}
\section{. Aufgabe}
\noindent
Die Ergebnisse sind in der anliegenden Datei aufgabe5.py implementiert.
Die Aufgabenteile a) bis d) basieren auf der Transformation der Gleichverteilung:
Dafür wird die normierte Wahrscheinlichkeitsdichte der Verteilungen aufgestellt,
welche gleich der gleichverteilten Zufallsvariablen ist. Durch invertieren der Dichte
wird auf die Zufallsvariable mit gewünschter Verteilung geschlossen.\\

\noindent
a) Normierung:
\begin{equation*}
  1 = \mathup{N}\,\int_{x_{\mathup{min}}}^{x_{\mathup{max}}}\,f(x)\,\mathup{d}x
\end{equation*}
$f(x)$ ist im Intervall $x_\mathup{min}$ bis $x_\mathup{max}$ gleich 1.
\begin{equation*}
  \mathup{N} = \frac{1}{x_\mathup{max}-x_\mathup{min}}
\end{equation*}
Die Wahrscheinlichkeitsdichte mit der gleichverteilten Zufallsvariablen $u$:
\begin{align*}
  \mathup{F}(x) &= u = \int_{x_{\mathup{min}}}^{x}\,\mathup{N}\,f(x)\mathup{d}x \\
  u &= \frac{x-x_{\mathup{min}}}{x_\mathup{max}-x_\mathup{min}}
\end{align*}
Invertieren:
\begin{equation*}
  x = u(x_\mathup{max}-x_\mathup{min})\,+\,x_\mathup{min}
\end{equation*}\\

\noindent
b) Normierung:
\begin{align*}
  1 &= \mathup{N}\,\int_{0}^{\infty}\,\mathup{e}^{-\frac{t}{\mathnormal{\tau}}}\,\mathup{d}t\\
  \mathup{N} &= 1\,/\,\mathnormal{\tau}
\end{align*}
Die Wahrscheinlichkeitsdichte mit der gleichverteilten Zufallsvariablen $u$:
\begin{align*}
  \mathup{F}(x) &= u = \int_{0}^{t}\,\mathup{N}\,\mathup{e}^{-\frac{t}{\mathnormal{\tau}}}\,{d}t \\
  u &= 1\,-\,\mathup{e}^{-\frac{t}{\mathnormal{\tau}}}
\end{align*}
Invertieren:
\begin{align*}
  t = -\mathnormal{\tau}\,\mathup{ln}(1-u)\\
\end{align*}

\noindent
c) Normierung:
\begin{align*}
  1 &= \mathup{N}\,\int_{x_{\mathup{min}}}^{x_{\mathup{max}}}x^{-\mathup{n}}\,\mathup{d}x\\
  \mathup{N} &= \frac{1-\mathup{n}}{x_\mathup{max}^{1-\mathup{n}}-x_\mathup{min}^{1-\mathup{n}}}
\end{align*}
Die Wahrscheinlichkeitsdichte mit der gleichverteilten Zufallsvariablen $u$:
\begin{align*}
  \mathup{F}(x) &= u = \int_{x_{\mathup{min}}}^{x}\,\mathup{N}\,x^{-\mathup{n}}\,{d}x \\
  u &= \frac{x^{1-\mathup{n}}-x_\mathup{min}^{1-\mathup{n}}}{x_\mathup{max}^{1-\mathup{n}}-x_\mathup{min}^{1-\mathup{n}}}
\end{align*}
Invertieren:
\begin{align*}
  x = (u\,(x_\mathup{max}^{1-\mathup{n}}-x_\mathup{min}^{1-\mathup{n}})\,+\,x_{\mathup{min}}^{1-\mathup{n}})^{(\frac{1}{1-\mathup{n}})}
\end{align*}

\noindent
d) Normierung:
\begin{align*}
  1 &= \mathup{N}\,\int_{x_{\mathup{min}}}^{x_{\mathup{max}}}\frac{1}{1+x²}\,\mathup{d}x\\
  \mathup{N} &= \frac{1}{\arctan{x_\mathup{max}}-\arctan{x_\mathup{min}}}
\end{align*}
N ist gleich $1/\pi$ für $x_\mathup{min} = -\infty$ und $x_\mathup{max} = \infty$.
Die Wahrscheinlichkeitsdichte mit der gleichverteilten Zufallsvariablen $u$:
\begin{align*}
  \mathup{F}(x) &= u = \int_{x_{\mathup{min}}}^{x}\,\mathup{N}\,\frac{1}{1+x²}\,{d}x \\
  u &= \frac{\arctan{x}-\arctan{x_\mathup{min}}}{\arctan{x_\mathup{max}}-\arctan{x_\mathup{min}}}
\end{align*}
Invertieren:
\begin{align*}
  x =  \tan{(u(\arctan{x_\mathup{max}}-\arctan{x_\mathup{min}})\,+\,\arctan{x_\mathup{min}})}
\end{align*}

\noindent
d) In diesem Aufgabenteil ist es notwendig das Neumann'sche Rückweisungsverfahren zu verwenden.
Ich habe es aber leider nicht geschafft die Aufgabe weiter zu bearbeiten :( .

\end{document}
